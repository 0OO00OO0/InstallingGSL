\documentclass{ctexart}

\usepackage{graphicx}

\usepackage{subfigure}

\usepackage{amsmath}

\usepackage{amsthm}
\newtheorem{property}{性质}

\usepackage{amssymb}

\usepackage{ulem}

\usepackage{url}

\usepackage{hyperref}
\hypersetup{hidelinks,
			colorlinks=true,
			allcolors=blue,
			pdfstartview=Fit,
			breaklinks=true}

\usepackage{geometry}
\geometry{left=2cm,right=2cm}

\title{GNU Scientific Library的安装及roots.c的功能}

\author{赵天健 \\ 信息与计算科学\quad 3210101830}

\begin{document}

\maketitle

\section{安装}
我用\verb!brew install gsl!直接安装了最新的\verb!2.7.1!版本的GSL, 可以在\verb!log/ChangeLog.log!中查看我安装的GSL的\verb!ChangeLog!文件.

\section{roots.c}
运行\verb!roots.c!文件, 会产生如下结果:
\begin{verbatim}
$ ./bin/roots
using brent method
 iter [    lower,     upper]      root        err  err(est)
    1 [1.0000000, 5.0000000] 1.0000000 -1.2360680 4.0000000
    2 [1.0000000, 3.0000000] 3.0000000 +0.7639320 2.0000000
    3 [2.0000000, 3.0000000] 2.0000000 -0.2360680 1.0000000
    4 [2.2000000, 3.0000000] 2.2000000 -0.0360680 0.8000000
    5 [2.2000000, 2.2366300] 2.2366300 +0.0005621 0.0366300
Converged:
    6 [2.2360634, 2.2366300] 2.2360634 -0.0000046 0.0005666
\end{verbatim}
\par \verb!roots.c!是一个求二次函数的根的程序, 可以通过\verb!gsl_roots.h!库中规定的一些迭代方法, 如Brent法, 二分法, Newton法( 需要额外给出函数的导函数 )等, 来求解二次函数$f(x)=x^2-5$在$(0, 5)$上的根, \verb!root!表示求得的近似根, \verb![lower, upper]!表示根落在的区间, \verb!err!表示实际误差, \verb!err(est)!表示最大可能误差( 即前述区间的长度 ). 若需要求解其它函数, 由一点数学可得需要找到使得函数取值符号相反的两点, 然后在此区间内使用GSL中的迭代方法即可求解.


\end{document}